\documentclass{article}
\usepackage[utf8]{inputenc}
\usepackage{mwe}
\usepackage{float}

% Math
\usepackage{mathtools}
\usepackage{amsfonts}

\usepackage{graphicx}
\usepackage{fancyhdr, tikz}
\usepackage{bm}
\usepackage[justification=centering]{caption}
\usepackage[position=bottom]{subfig}

% Geometry
\usepackage{geometry}
\geometry{top=20mm}
\geometry{bottom=20mm}

\usepackage{hyperref}
\usepackage[toc,page]{appendix}
\usepackage{lipsum}
\usepackage{titlesec}
\usepackage{caption}
\usepackage{scrextend}
\usepackage{framed}
\captionsetup{compatibility=false}

\title{Project in Solid Mechanics}
\author{Vilmer Dahlberg}
\date{\today}
% CA
\renewcommand{\vec}[1]{\mathbf{#1}}
\renewcommand{\u}{\vec{u}}
\newcommand{\tu}{\Tilde{\u}}
\newcommand{\pluseq}{\mathrel{+}=}
\newcommand{\mineq}{\mathrel{-}=}
\newcommand{\asteq}{\mathrel{*}=}

\newcommand{\K}{\vec{K}}
\newcommand{\delK}{\Delta\vec{K}}
\newcommand{\B}{\vec{B}}
\newcommand{\Rb}{\vec{R_b}}
\newcommand{\Vb}{\vec{V_b}}
\newcommand{\phat}{\hat{\bm\phi}}

\newcommand{\f}{\vec{f}}
\renewcommand{\i}{^{-1}}

% Nonlinear FEM
\newcommand{\veca}{\vec{a}}
\newcommand{\dela}{\Delta \veca}

\newcommand{\fint}{\boldsymbol{f}_{int}}
\newcommand{\fintf}{\boldsymbol{f}_{int, f}}
\newcommand{\fintp}{\boldsymbol{f}_{int, p}}
\newcommand{\fext}{\boldsymbol{f}_{ext}}
\newcommand{\fextf}{\boldsymbol{f}_{ext, f}}
\newcommand{\fextp}{\boldsymbol{f}_{ext, p}}
\renewcommand{\r}{\vec{r}}
\newcommand{\zero}{\vec{0}}
\newcommand{\z}{\vec{z}}


\begin{document}
\section{Combined approximation}
The goal is to approximate $\u$ given $\K \u = \f$.

A reduced basis is generated using a previous factorization of $\K _0$. Let $\K = \K_0 + \delK$, then the equilibrium equations can be written as
\begin{equation}
    (\K _0 + \delK)\u = \f.
\end{equation}
A recurrence equation can be found by moving terms to the right-hand side
\begin{equation}
    \K _0 \u_{k + 1} = \f - \delK \u_k.
\end{equation}
Substitution gives $\u_{k + 1}$ in terms of $\u_1 = \K\i_0 \f$
\begin{equation}
    \u_{k + 1} = \bigg( \bm{I + \B + \B^2 + ... + \B^k}\bigg)\u_1, \ \text{where} \ \B = -\K\i_0 \delK.
\end{equation}
Thus, $\u$ can be expressed in the basis defined by
\[
\begin{cases}
    \bm \phi_1 = \K\i_0 \f \\
    \bm \phi_k = \K\i_0 \bm \delK \phi_{k-1}.
\end{cases}
\]
If the change in the matrix $\K$ is small (given by $\delK$), generating a small amount of basis vectors can yield an accurate approximation. Let $\Rb$ be the matrix consisting of the s first basis vectors, meaning $\Rb = (\bm{\phi_1, \; ..., \; \phi_s})$. The approximation can be expressed as $\Rb \Vec{y} = \tu$. Substitution into the equilibrium equation, and premultiplying by $\Rb^T$ from the left gives a reduces system of equations
\begin{equation}
    \Rb ^T \K \Rb \Vec{y} = \Rb^T \f.
\end{equation}

Since the basis vectors $\bm \phi_k$ may be almost parallel, orthogonalizing them can make the scheme more numerically stable. Furthermore, orthogonalizing with respect to the matrix $\K$ (meaning $\phat_i^T \K \phat_j = \delta_{ij}$) makes the system very easy to solve. Let $\Vb$ denote the orthogonal basis generated by the Gram-Schmidt scheme. The resulting system is given by
$$
\Vb ^T \K \Vb \Vec{z} = \Vb^T \f.
$$
Due to the orthogonalization the left-hand side reduces to the identity matrix, $\Vb ^T \K \Vb = \bm I$, and the approximation is given by
$$
\tu = \Vb \Vec{z} = \Vb \Vb^T \f.
$$

% Jämför med numerisk känslighet
% approximativ känslighet
% Kolla om materialmodellen funkar som det ska
% Eventuellt öka volymhalten
% Enkel approximation till känsligheten
% Oded

\section{Nonlinear FEM}
The displacements $\veca$ are found by solving the nonlinear system of equations
$$
\r = \fint(\z, \veca) - \fext = \zero.
$$
The stresses in the structure are given by the change in the strain energy with respect to the strain, or
$$
\vec{S} = \frac{\partial w}{\partial \vec{E}}.
$$
Discretizing the structure using finite elements, the internal forces in the structure are given by
$$
\fint = \int_\Omega \B^T \vec{S}dv.
$$
A Newton-Raphson scheme is used to iteratively solve for the displacements $\veca$,
$$
\K \dela_{k + 1}= -\r_{k}
$$
The displacements are then updated $\veca_{k + 1} = \veca_k + \dela_{k + 1}$.

\section{Line search}
It can be shown that the number of iterations in the solution scheme can be reduced by introducing line search. Instead of updating the displacements directly using the update $\bm a^i = \bm a^{i - 1} + \bm \Delta \bm a$, the increment in displacements is used as a search direction to find the next displacements $\bm a^i = \bm a^{i - 1} + \beta \bm \Delta \bm a$. Introduce the quantity $r(\beta) = \bm \Delta \bm a^T\bm r(\bm a^{i - 1} + \beta\bm \Delta \bm a)$. The residual is minimized if $\beta$ is chosen as
$$
\hat{\beta} = \frac{r(0)}{r(0) + r(1)}.
$$
Too large or small values of $\beta$ may cause issues, so in practice $\beta$ is chosen as
$$
\beta = \max\left(3, \ \min\left(0.3, \ \hat{\beta}\right)\right).
$$

\section{Optimization}
The end compliance can be used as a measure of the stiffness of the structure. If a displacement controlled scheme is used, maximizing the compliance results in a stiff structure.
$$
g_o = c = \bm \u_p^T\fintp.
$$
The design variables $\bm z$ are filtered using a density filter to prevent a checkerboard pattern, meaning $\z_f = \vec{M}_f\z$.
% The filtered design variables are projected onto [0, 1] using a Heaviside function which gives the densities $\bm \rho = H(\z_f)$.
A SIMP interpolation scheme gives elasticity modulus 
$$
E_e(\rho_e) = E_{min} + (E_{max} - E_{min})\rho_e^q,
$$
where q is usually chosen as 3.

A Neo-Hookean material model is used to model the structure, meaning the strain energy is given by
$$
w^{NH} = \frac{\kappa}{2}\bigg(\frac{1}{2}(J^2 - 1) - ln(J)\bigg) + \frac{1}{2}\mu \bigg(J^{-2/3}tr(\Vec{C}) - 3\bigg),
$$
where $J = \text{det}(\vec{F})$, $\vec{C}$ is the Cauchy-Green tensor $\vec{C} = \vec{F}^T \vec{F}$, $\kappa  = \frac{E}{3(1 - 2\nu)}$ is the bulk modulus, and $\mu = \frac{E}{2(1 + \nu)}$ is the shear modulus.

The strain energy is linear in the elasticity modulus $E_e(\rho_e)$, which is independent on the strain $\bm E$. This gives a very easy connection between the design variables and the internal forces, which can be evaluated choosing E = 1, and then multiplying by the interpolated elasticity modulus.
$$
(\fint)_e = E_e(\rho_e) \int_{V_e} \B^T \bm S^{NH}_e\big |_{E = 1}dV_e.
$$
The sensitivity of the forces with respect to the element density is thus
$$
\frac{d\boldsymbol{f}_{int, e}}{d\rho_e} = \frac{dE_e(\rho_e)}{d\rho_e}\int_{V_e} \B^T \bm S^{NH}_e\big |_{E = 1}dV_e.
$$
\subsection{Sensitivities}
Instead of computing the sensitivity of the goal function, an augmented goal function is analysed in order to reduce computational cost. Thus, consider the function
$$
\Tilde{g_0} = \bm \u_p^T\fintp + \bm \lambda^T\fintf.
$$
Assuming the structure is in equilibrium, this augmented goal function is identical to the original goal function. The sensitivity is given by
\begin{align*}
\frac{d \Tilde{g_0}}{dz_e} & = \u_p^T\left(\frac{\partial \fintp}{\partial z_e} + \frac{\partial \fintp}{\partial \u_f}\frac{\partial \u_f}{\partial z_e}\right) + \bm \lambda^T\bigg(\frac{d \fintf}{dz_e} + \frac{d \fintf}{d\bm \u_f}\frac{d\bm \u_f}{dz_e}\bigg) = \\ 
& = \u_p^T\frac{\partial \fintp}{\partial z_e} + \bm \lambda^T \frac{\partial \fintp}{\partial z_e} + \left(\bm \lambda^T\frac{d\fintf}{d\bm \u_p} + \u_p^T \frac{\partial \fintp}{\partial \u_f}\right)\frac{d\bm \u_f}{dz_e}.
\end{align*}
The derivative $\frac{d\bm \u_f}{dz_e}$ can be eliminated by asserting 
$$
\bm \lambda^T\frac{d\fintf}{d\bm \u_p} + \u_p \frac{\partial \fintp}{\partial \u_f} = \bm 0
$$
Using the notation $\frac{d\fintf}{d\u_f} = \bm K_{ff}$, and $\frac{d\fintp}{d\u_f} = \bm K_{fp}$ we have
\begin{align}
    & \frac{d\Tilde{g_0}}{dz_e} = \frac{d\fintf^T}{dz_e}\bm \lambda \nonumber \\
    & \bm K_{ff}\bm \lambda = -\bm K_{fp}\u_p. \label{eq:sens}
\end{align}
\section{Reusing information}
The idea of reusing information between optimization steps is what \textit{Combined Approximation} is based on. \textit{Combined Approximation} reuses information of the stiffness matrix from the previous step to speed up the solution of the non-linear system of residual equations, which is the most costly step in the optimization scheme as a whole. However, the change in design variables can also be used to find an initial guess to the increment in displacement. Expanding the residual equations in $\bm z$ and $\bm u$ we have
$$
\bm r_{f}(\bm z^{i}, \bm u^{i}) \approx \bm r_{f}(\bm z^{i-1}, \bm u^{i - 1}) + \frac{\partial \bm r^{i-1}_{f}}{\partial \bm z}\bm \Delta \bm z + \frac{\partial \bm r^{i-1}_{f}}{\partial \bm u_f}\bm \Delta \bm u_f.
$$
Assuming the residual equations are solved at the previous step $\bm r^{i-1} = \bm 0$, and asserting the residual equations are solved at the current step $\bm r^{i} = \bm 0$, an initial guess for the displacements can be found
$$
\frac{\partial \bm r^{i-1}_{f}}{\partial \bm u_f}\bm \Delta \bm u_f = -\frac{\partial \bm r^{i-1}_{f}}{\partial \bm z}\bm \Delta \bm z.
$$
Since the value of $\frac{\partial \bm r^{i-1}_{f}}{\partial \bm z}$ was computed in the evaluation of the sensitivities in the previous step, and the matrix $\frac{\partial \bm r^{i-1}_{f}}{\partial \bm u_f}$ has already been factorized and used in the iterative scheme, (or CA can be used to find a good approximation) this is very cheap to evaluate.
As we have discarded higher order terms, the accuracy depends on higher order derivatives and the the norm of $\bm z$, so in practice this initial guess of the displacement increment is only made if the norm of the change in design is small enough.
\section{Pseudocode}
\begin{minipage}[t]{0.7\textwidth}
\begin{figure}[H]
\begin{framed}
\textit{Optimization scheme}
\hrule
\begin{itemize}
    \item[] \textbf{for} i = 1  \textbf{to} maxits
    \begin{itemize}
        \item[-] Solve for displacements and forces using NR
        \begin{itemize}
            \item[] \textbf{if} i == 1
            \begin{itemize}
                \item[-] take \textit{n} steps
            \end{itemize}
            \item[] \textbf{else}
            \begin{itemize}
                \item[-] take 1 step with initial guess $\bm u_{i - 1}^{n}, \ \text{and} \ \bm K_{i - 1}^{n}$
            \end{itemize} 
            \item[] \textbf{end if}
        \end{itemize}
        \item[-] Compute sensitivities, solving equation \ref{eq:sens} with CA
        \item[-] Solve mma-subproblem
    \end{itemize}
    \item[] \textbf{end for}
\end{itemize}
\end{framed}
\caption{The complete optimzation scheme for the nonlinear optimization.}
\end{figure}
\end{minipage}\\

\begin{minipage}[t]{0.7\textwidth}
\begin{figure}[H]
\begin{framed}
    \textit{Combined Approximation}
    \hrule
    \begin{itemize}
        \item[-] Initialize quantities
        \item[-] $\bm r_1 = \K\i_0\f, \ \delK = \bm K - \bm K_0$
        \item[-] $\bm v_1 = (\bm r_1^T\K \bm r_1)^{-1/2}\bm r_1$
        \item[] \textbf{for} k = 2 \textbf{to} s
        \begin{itemize}
            \item[-] compute basis $\bm r_k = -\K\i_0 \delK \bm r_{k - 1}$
            \item[-] initialize orthogonalization $\bm v_k = \bm r_k$
            \item[] \textbf{for} j = 1 \textbf{to} k - 1
            \begin{itemize}
                \item[-] $\bm v_k \mineq (\bm r_k^T\K \bm v_j)\bm v_j$
            \end{itemize}
            \item[] \textbf{end for}
            \item[-] normalize $\bm v_k \asteq (\bm v_k^T\K \bm v_k)^{-1/2}$
        \end{itemize}
        \item[] \textbf{end for}
        \item[-] Solve $\bm u = \bm V(\bm V^T \bm f)$
    \end{itemize}

\end{framed}
\label{fig:pseudocodeGS}
\caption{Pseudocode for the combined approximation. It's very simple.}
\end{figure}
\end{minipage}\\

\begin{minipage}[t]{0.7\textwidth}
\centering
\begin{figure}[H]
\begin{framed}
\textit{Newton-Raphson (Displacement Controlled)}
\hrule
\vspace{1ex}
Given design $\bm z_i =\bm z_{i-1} + \bm \Delta \bm z$ at optimization step \textit{i}:
\begin{itemize}
    \item[-] Initialize quantities
    \item[] \textbf{for} k = 1 \textbf{to }\# of loadsteps
    \begin{itemize}
        \item[-] Update displacement vector $\bm u_i^k = \bm u_i^{k-1} + \bm \Delta \bm u$
        \item[] \textbf{do while} $||\bm r_{free}|| > r_{tol}$
        \begin{itemize}
            \item[-] Compute stiffness matrix $\bm K(\bm u_i^k, \bm z_i) = \bm K^k_i$
            \item[] \textbf{if} $\cos(\theta(\bm z_i,\bm z_{i-1})) > c_{tol}$
            \begin{itemize}
                \item[-] fetch $\bm K^k_{i-1}, \ \text{and}\ \bm R^k_{i-1}$
                \item[-] solve $\bm K^k_i \bm s = -\bm r$ with CA
            \end{itemize}
            \item[] \textbf{else}
                \begin{itemize}
                    \item[-] factorize $\bm K^k_i = \left(\bm R^k_i\right)^T\bm R^k_i$
                    \item[-] solve $\bm s = -\bm R^k_i \Big \backslash \left(\bm R^k_i\right)^T\Big \backslash \bm r$
                    \item[-] save $\bm R^k_i$
                \end{itemize}
            \item[] \textbf{end if}
            \item[-] Search for $\bm u_i^k$ along the direction $\bm s$
            \item[-] Update residual $\bm r = \bm r(\bm u^k_i, \bm z_i)$
        \end{itemize}
        \item[] \textbf{end while}
        \item[-] Save $\bm K^k_i$ and $\bm u_i^k$.
    \end{itemize}
    \item[] \textbf{end for} 
\end{itemize}
\end{framed}
\label{}
\caption{Newton Raphson scheme using the combined approximation for solving the linear equation system.}
\end{figure}
\end{minipage}\\
\begin{minipage}[y]{0.7\textwidth}
Both changes is design and changes in the displacements perturb $\bm K$ and can result in poor performance for CA if the perturbation in $\bm K$ is large. A check for the change in displacements can be used in tandem with the design change check to ensure CA performs well (although I have not found time to test such an implementation).
\end{minipage}

\end{document}
